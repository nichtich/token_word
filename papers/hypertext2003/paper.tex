%
% Modification History
%
% 2003-February-6  Jason Rohrer
% Created.
%


\documentclass{acm_proc_article-sp}

\newcommand{\tw}{token\_word}

\begin{document}

\title{\tw:  Building a Xanalogical System in Nine Days}

\numberofauthors{1}

\author{
\alignauthor Jason Rohrer\\
    \affaddr{University of California, Santa Cruz}\\
    \affaddr{Dept. of Computer Science}\\
    \affaddr{Santa Cruz, CA 95064}\\
    \email{rohrer@cse.ucsc.edu}
}

\date{February 6, 2003}


\maketitle

\begin{abstract}
We describe \tw, a web-based xanalogical hypertext system.  Our system supports almost every feature associated with Xanadu, an architecture that has been actively discussed for nearly forty years.
Features include deep quotation, unbreakable references, two-way reference chasing, and frictionless micropayments backed by real-world currency.
To our knowledge, \tw \ is the first xanalogical system implementation to included all of these features.  Also, \tw \ is the first xanalogical system of any kind to be accessible through a web browser with no additional software.

Though previous attempts at xanalogical implementations took many years and ultimately were never completed, \tw \ was implented from scratch in nine days by one person.  
We attribute this success to the maturity of modern technology and design techniques.  
In particular, choosing Perl as the implementation language allowed the rather complex xanalogical text processing operations to be expressed in elegant ways.

\end{abstract}

\category{H.5.4}{Information Interfaces and Applications}{Hypertext/Hypermedia}[Architectures]
\category{K.4}{Computers and Society}{Electronic Commerce}[Payment Schemes]


\terms{Design, Human Factors}

\keywords{Micropayments, quotation, transclusion, Xanadu}

\section{Introduction}
Xanadu has been floating in the collective mind of the hypertext research community for nearly forty years.
Xanadu is a dream and an ideal.  
We know it would be great (or at least interesting) if a xanalogical system existed, but we are also resigned to the notion that such a system will probably never exist.
Is Xanadu impossible?

That question by itself, the impossibility question, is extraordinarily intriguing.  We {\it know} that Xanadu is not impossible.  
The design, on paper, is clear enough:  we just need to build it.  
The problem might be that no one tried to build it.  
But many people did try to build Xanadu over the years.  
The problem might be that the people trying to build it did not have sufficient resources.  But people with extraordinary resources did try to build it.  
Will anyone argue that Autodesk did not have extraordinary software development resources in the late 1980s?  
Furthermore, Autodesk {\it believed} in Xanadu.  
The following quote from a 1988 Autodesk press release is most telling:

\begin{quote}
In 1964, Xanadu was a dream in a single mind.  In 1980, it was the
shared goal of a small group of brilliant technologists.  By 1989, it
will be a product.  And by 1995 it will begin to change the world.
Much work remains to be done to realise the potential of Xanadu--it
will take the Xanadu development team 18 months to field the first
Xanadu system. \cite{AutodeskPress} 
\end{quote}

In that same press release, Autodesk claims 1987 sales figures topping \$79 million.  
Indeed, Autodesk had extraordinary resources for its time.  
In 1992, after four years of development and \$5 million spent \cite{AutodeskCost},  Autodesk ceased development of Xanadu \cite{AutodeskPressDrop}.  
Ironically, this decision was made just when the world was ripe for its first global hypertext system \cite{BernersLee92}.

We are scared of Xanadu.  
We hear its siren song and are tempted, but we know that many have tried and many have failed.  
The most frustrating factor is that none of us really know how useful Xanadu would be, even if it {\it did} exist.  
We can look at architecture drawings and pictures of cardboard mock-ups\cite{Nelson1999}, but these do not give us any sense of what it would be like to use a xanalogical system.  
Few of us have ever played with a xanalogical software prototype, even fewer have paid or received a true micropayment (xanalogical or otherwise), and almost none have transcluded the work of another or had our work transcluded.

What is using a xanalogical system really like?
Is it as useful as it sounds?  
This paper attempts to answer this question by describing the first fully-realized xanalogical implementation.


\section{The Xanadu Model}

As mentioned in the above quote, Xanadu's underlying ideas were first incubated by Nelson in the 1960s \cite{Nelson1965}.  
These ideas evolved somewhat over the next forty years, receiving one of their most verbose presentations in Nelson's book {\it Literary Machines}, which was first published in 1981 \cite{NelsonLiteraryMachines}.  
Nelson's more recent paper on the xanalogical model is up to date in terms of modern technological trends, but the core ideas have not changed \cite{Nelson1999}.  
The high-level description and interpretation of Xanadu given here is based on information culled from {\it Literary Machines}, Nelson's recent lectures, and personal conversations with Nelson.

The goal of Xanadu, distilled into a single phrase, is {\it frictionless content reuse}.  
This goal would be easily realized in a frictionless, copyright-free world:  authors could simply copy whatever they want and reuse it however they want.  
In our world, however, we have copyright.
Unlike other systems \cite{Clark2000}, Xanadu does not attempt to undermine copyright, but instead tries to work with it. 
All of the design choices and complications in the Xanadu model can be seen as mechanisms for making reuse frictionless in the face of copyright.
When trying to implement unrestricted reuse in a way that works with copyright, we are faced with the following fact:  {\it copyright forbids unrestricted copying}.

Xanadu's central mechanism is {\it transclusion}, a form of quotation.  
Transclusion differs from quotation in that it does not involve copying of any kind.  
When document $A$ transcludes a quote from document $B$ the quote is stored as a deep reference to the quoted material in $B$:  words from $B$ are not copied into $A$. 
Whenver $A$ is viewed, the words from $A$ are fetched from $A$'s owner, and the necessary word from $B$ are fetched from $B$'s owner.
Thus, even though $A$'s owner has quoted document $B$, the quoted owner still retains control over his or her words.
Avoiding copying led us to transclusion, and making transclusion work will lead us to the rest of the Xanadu model.

Transclusion creates a serious problem, at least in combination with how we commonly think of electronic documents.
Electronic documents change constantly.
If $B$ changes or is deleted, $A$'s quote can break.
Xanadu's solution is to forbid document change but encourage document versioning.
Thus, as $B$ changes, all previous versions are maintained, and $A$'s quote can point to the particular version that $A$'s owner quoted.
A particular version of $B$ is immutable, so $A$'s quote will never break.

The Xanadu model refines this versioning model further by describing an ever-growing, add-only, content space.  
When a document is created, its text is added to this space.
When a document quotes this document, it points not to the document itself, but to the quoted words in the content space.
Thus, a document in Xanadu is a series of pointers to regions in the content space, and no text is contained in the documents themselves.
Versioning with this mechanism is particularly space-efficient:  unchanged portions of the document can be quoted from the content space instead of being duplicated.



\section{System Architecture}


\subsection{Xanalogical Improvements}


\subsection{Missing Features}


\section{Implementation}


\section{Micropayment Issues}


\subsection{Balancing Risk}



\section{Related work}

Wikis
\cite{WikiWikiWeb}
\cite{Wikipedia}

Everything2
\cite{Everything2}

Weblogs
\cite{kuro5hin}

Xanadu

\section{Conclusion}


\section{Acknowledgements}
Thanks are due to Ted Nelson and Jim Whitehead for their contributions to interesting xanalogical discussions over the past year. ``Xanadu'' is a trademark of Ted Nelson.  To prevent confusion, the generic terms ({\it e.g.}, xanalogical) have been used wherever possible.

\bibliographystyle{abbrv}
\bibliography{paper}
\end{document}